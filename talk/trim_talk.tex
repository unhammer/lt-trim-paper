% Created 2009-05-08 Fri 15:22
\documentclass[notes=hide]{beamer}
\usepackage[utf8]{inputenc}
\usepackage[T1]{fontenc}
\usepackage{hyperref}
\usepackage[english]{babel}
\usepackage{apacite} % after babel
\usepackage{natbib}
\usepackage{pslatex}
\pdfoutput=1

\usepackage{enumerate}
\usepackage{subfigure}
\usepackage{linguex}

\usepackage{color}
\usepackage{xcolor}

\usepackage{graphicx}       % for \includegraphics, only works when {cslipubs} is [FINAL]
\usepackage{amssymb}        % for \square

\usetheme{Goettingen}
%\setbeameroption{show notes}

\newcommand{\ana}[1]{\texttt{#1}}
\newcommand{\form}[1]{`#1'}
\newcommand{\tool}[1]{\texttt{#1}}
\newcommand{\fn}[1]{$\mathrm{#1}$}

\title[lt-trim]{FST Trimming: Ending Dictionary Redundancy in Apertium}
\author{Matthew Marting\inst{1} \and Kevin Brubeck Unhammer\inst{2}}
\date{27th May 2014}
\institute[apertium]{
  \inst{1} St. David's School \\ Raleigh, NC. \\ {\tt \tiny $\emptyset{}$}
  \and
  \inst{2} Kaldera språkteknologi \\ Stavanger, Norway \\ {\tt \tiny unhammer+apertium@mm.st}
}

\begin{document}
\maketitle


\begin{frame}
  \frametitle{Outline of talk}
  \note{}
\setcounter{tocdepth}{1}
\tableofcontents[] % add pausesections? (one slide per \section)
\setcounter{tocdepth}{3}
\end{frame}

\section{Introduction and background}
\begin{frame}\frametitle{Introduction and background}
  \note{}
  \begin{itemize}
    \item
  \end{itemize}
\end{frame}

\subsection{FST's in the Apertium pipeline}
\begin{frame}
  \frametitle{FST's in the Apertium pipeline}
  \begin{itemize}
  \item
  \end{itemize}
\end{frame}


\subsection{The Problem: Redundant data}
\begin{frame}
  \frametitle{The Problem: Redundant data}

  \begin{figure}[h]
    \begin{center}
      \includegraphics[scale=0.6]{../pairs-before.eps}
      \caption{Current number of monodixes with pairs of four languages}
      \label{fig:monodixes-current}
    \end{center}
  \end{figure}

  \begin{figure}[h]
    \begin{center}
      \includegraphics[scale=0.6]{../pairs-after.eps}
      \caption{Ideal number of monodixes with four languages}
      \label{fig:monodixes-ideally}
    \end{center}
  \end{figure}

\end{frame}

\subsection{A Solution: Intersection}
\begin{frame}
  \frametitle{A Solution: Intersection}
  \begin{itemize}
  \item
  \end{itemize}
\end{frame}

\section{Implementation of \tool{lt-trim}}
\begin{frame}
  \frametitle{Implementation of \tool{lt-trim}}
  \begin{itemize}
  \item
  \end{itemize}
\end{frame}

\subsection{Preprocessing the bilingual dictionary}
\begin{frame}
  \frametitle{Preprocessing the bilingual dictionary}
  \begin{itemize}
  \item
  \end{itemize}
\end{frame}

\subsection{Prefixing the bilingual dictionary}
\begin{frame}
  \frametitle{Prefixing the bilingual dictionary}
  \begin{itemize}
  \item
  \end{itemize}
\end{frame}

\subsection{Moving uninflected lemma parts}
\begin{frame}
  \frametitle{Moving uninflected lemma parts}

\begin{figure}[h]
  \begin{center}
    \includegraphics[scale=0.7]{../bi.eps}
    \caption{Input bilingual FST (letter transitions compressed to single arcs)}
    \label{fig:bi-prefixed}
  \end{center}
\end{figure}

\begin{figure}[h]
  \begin{center}
    \includegraphics[scale=0.7]{../bi-prefixed.eps}
    \caption{Fully preprocessed bilingual FST; analyses \ana{take<vblex>\# out} and even \ana{take<vblex>+it<prn>\# out}    would be included after trimming on this}
    \label{fig:bi-prefixed}
  \end{center}
\end{figure}

\end{frame}

\subsection{Intersection}
\begin{frame}
  \frametitle{Intersection}

  \begin{figure}[h]
    \begin{center}
      \includegraphics[scale=0.7]{../mono.eps}
      \caption{Input monolingual FST}
      \label{fig:mono}
    \end{center}
  \end{figure}

  \begin{figure}[h]
    \begin{center}
      \includegraphics[scale=0.7]{../trimmed.eps}
      \caption{Trimmed monolingual FST}
      \label{fig:trimmed}
    \end{center}
  \end{figure}

\end{frame}

\subsection{\tool{lt-trim} in use}
\begin{frame}
  \frametitle{\tool{lt-trim} in use}
  \begin{itemize}
  \item
  \end{itemize}
\end{frame}

\section{Ending Dictionary Redundancy}
\begin{frame}
  \frametitle{Ending Dictionary Redundancy}
  \begin{itemize}
  \item
  \end{itemize}
\end{frame}

\section{Conclusion}
\begin{frame}
  \frametitle{Conclusion}
  \begin{itemize}
  \item
  \end{itemize}
\end{frame}

\section*{Acknowledgements}
\begin{frame}
  \frametitle{Acknowledgements}
  \begin{itemize}
  \item
  \end{itemize}
\end{frame}



\end{document}
