% Created 2009-05-08 Fri 15:22
\documentclass[notes=hide]{beamer}
\usepackage[utf8]{inputenc}
\usepackage[T1]{fontenc}
\usepackage{hyperref}
\usepackage[english]{babel}
\usepackage{apacite} % after babel
\usepackage{natbib}
\usepackage{pslatex}
\pdfoutput=1

\usepackage{enumerate}
\usepackage{subfigure}
\usepackage{linguex}

\usepackage{color}
\usepackage{xcolor}

\usepackage{graphicx}       % for \includegraphics, only works when {cslipubs} is [FINAL]
\usepackage{amssymb}        % for \square

\usetheme{Goettingen}
%\setbeameroption{show notes}

\newcommand{\ana}[1]{\texttt{#1}}
\newcommand{\form}[1]{`#1'}
\newcommand{\tool}[1]{\texttt{#1}}
\newcommand{\fn}[1]{$\mathrm{#1}$}

\title[lt-trim]{FST Trimming: Ending Dictionary Redundancy in Apertium}
\author{Francis Tyers\inst{0} \and Matthew Marting\inst{1} \and Kevin Unhammer\inst{2}}
\date{27th May 2014}
\institute[apertium]{
  \inst{0} UiT Norgga árktalaš universitehta \\ Romssa, Norga \\ {\tt \tiny ftyers@prompsit.com}
  \and
  \inst{1} St. David's School \\ Raleigh, NC. \\ {\tt \tiny $\emptyset{}$}
  \and
  \inst{2} Kaldera språkteknologi \\ Stavanger, Noreg \\ {\tt \tiny unhammer+apertium@mm.st}
}

\begin{document}
\maketitle


\begin{frame}
  \frametitle{Outline of talk}
  \note{}
\setcounter{tocdepth}{1}
\tableofcontents[] % add pausesections? (one slide per \section)
\setcounter{tocdepth}{3}
\end{frame}

\section{Introduction and background}
\begin{frame}\frametitle{Introduction and background}
  \note{}
  \begin{itemize}
    \item Apertium: Free/Open Source, Rule-based Machine Translation platform
    \item Goals include:
      \begin{itemize}
      \item supporting lesser-resourced languages
      \item wide coverage
      \item post-editable output
      \item reusable resources
      \end{itemize}
    \item Language data (dictionaries, etc.) typically organised in
      language \emph{pairs} (Catalan-Spanish, Portuguese-Spanish,
      etc.)
      \begin{itemize}
      \item historically: each with its own copy of monolingual data
      \end{itemize}
  \end{itemize}
\end{frame}

\subsection{FST's in the Apertium pipeline}
\begin{frame}
  \frametitle{Apertium pipeline architecture}
  \begin{itemize}
    \item lttoolbox Finite State Transducers used for, among others:
      \begin{itemize}
      \item morph. analysis: \form{fishes} to \ana{fish<n><pl>/fish<vblex><pres>}
      \item lex. transfer: \ana{fish<n><pl>} to \ana{fisk<n><m><pl>}
      \item morph. generation: \ana{fisk<n><m><pl><def>} to \form{fiskane}
      \end{itemize}
  \end{itemize}
  \begin{figure}[h]
    \begin{center}
      \includegraphics[scale=0.8]{architecture.eps}
      %\caption{The Apertium pipeline}
      \label{fig:architecture}
    \end{center}
  \end{figure}
\end{frame}

\begin{frame}
  \frametitle{Multiword support}
  lttoolbox FST's support a variety of multiwords

  ~ \\

  An lttoolbox ``lexical unit'' is one token, and can be:
  \begin{itemize}
  \item simple non-multi-words: \form{fish}
  \item simple space-separated words: \form{hairy frogfish} as a
    single token
  \item multiwords with \textbf{inner inflection}: \form{takes out}, \\
    analysed as \ana{take<vblex><pri><p3><sg>\# out},\\
    converted to \ana{take\# out<vblex><pri><p3><sg>} before lexical
    transfer
  \end{itemize}
\end{frame}

\begin{frame}
  \frametitle{Multiword support}

  \begin{itemize}
  \item \textbf{joined} multiwords: \form{they'll}; \\
    analysed as single token \ana{prpers<prn><subj><p3><mf><pl>+will<vaux><inf>}, \\
    then split into two tokens \ana{prpers<prn><subj><p3><mf><pl>} and
    \ana{will<vaux><inf>} before lexical transfer
  \item \textbf{compounds}: \form{frogfish}; \\
    analysed as single token \ana{frog<n><sg>+fish<n><pl>}, \\
    then split into two tokens \ana{frog<n><sg>} and \ana{fish<n><pl>}
    before lexical transfer
  \end{itemize}
\end{frame}

\begin{frame}
  \frametitle{Multiword support}
  \begin{itemize}
  \item combinations (space-separated + joined + inner inflection):
    \form{creure-ho que},\\
    analysed as single token \ana{creure<vblex><inf>+ho<prn><enc><p3><nt>\# que}, \\
    then moved and split into two tokens \\
    \ana{creure\# que<vblex><inf>} and \ana{ho<prn><enc><p3><nt>}
    before lexical transfer
  \end{itemize}
\end{frame}

\subsection{The Problem: Redundant data}
\begin{frame}
  \frametitle{The Problem: Redundant data}

  \begin{figure}[h]
    \begin{center}
      \includegraphics[scale=0.6]{../pairs-after.eps}
      \caption{Ideal number of monodixes with four languages}
      \label{fig:monodixes-ideally}
    \end{center}
  \end{figure}

  \begin{figure}[h]
    \begin{center}
      \includegraphics[scale=0.6]{../pairs-before.eps}
      \caption{Current number of monodixes with pairs of four languages}
      \label{fig:monodixes-current}
    \end{center}
  \end{figure}

\end{frame}

\begin{frame}
  Words in analyser but missing from lexical transfer can be
  problematic:
  \begin{itemize}
  \item \form{fishes} to \form{@fish}: loses the inflection
  \item \form{gikk til hundene} ``went to the dogs'' to
    \form{went to @hund} ``went to dog'': losing the
    inflection hides the idiomatic meaning
  \item \form{öldürmedi} ``did not kill'' to \form{@öl} ``kill'':
    loses the \emph{negation}
  \item lexical transfer is also tag transfer – structural transfer
    thus needs exceptions for half-translated tags
  \end{itemize}
\end{frame}

\begin{frame}
  But, most importantly, multiword tokenisation means that

  ~

  \form{He takes out the trash} translates to \form{Han @take out
    søpla} \emph{even though both \form{take}-\form{ta} and
    \form{out}-\form{ut} are in the bilingual dictionary}.

  ~

  Adding more words makes the translator worse!
\end{frame}

\subsection{A Solution: Intersection}
\begin{frame}
  \frametitle{A Solution: Intersection}
  Compile a \emph{trimmed} analyser-FST containing only entries from
  monolingual FST that would pass through bilingual FST.

  ~\\

  Goal: One big monolingual dictionary, trimmed during compile to
  language-pair specific analysers.

  ~\\

  \begin{itemize}
  \item We know we can do: FSA1 $\cap$ FSA2 = FSA3
  \item We want: output of FST1 $\cap$ input of FST2 = FST3
  \end{itemize}

\end{frame}

\begin{frame}
  With some exceptions:
  \begin{itemize}
    \item First need to append $.*$ to FST2 \\(because lexical transfer only needs
      a match on the start of the string)
    \item And reorder \#-multiwords in FST2 \\(so they look
      like FST1, otherwise they won't match)
    \item And let a + in FST1 mean transition-to-start in FST2 \\(since
      single token \ana{a+b} in FST1 is split into two tokens \ana{a} \ana{b} before
      lexical transfer)
  \end{itemize}
\end{frame}

\section{Implementation of \tool{lt-trim}}
\begin{frame}
  \frametitle{Implementation of \tool{lt-trim}}
  Tool takes two compiled FST's, produces a new, \emph{trimmed} FST
  \begin{itemize}
  \item Preprocess bilingual FST
    \begin{itemize}
    \item Append $.*$
    \item Reorder \#-multiwords
    \end{itemize}
  \item Depth-first intersection of output of FST1 with input of FST2
    \begin{itemize}
    \item with an exception on seeing +
    \end{itemize}
  \end{itemize}
\end{frame}

\subsection{Preprocessing the bilingual dictionary}
\begin{frame}
  \frametitle{Preprocessing the bilingual dictionary}
  \begin{itemize}
  \item
  \end{itemize}
\end{frame}

\subsection{Prefixing the bilingual dictionary}
\begin{frame}
  \frametitle{Prefixing the bilingual dictionary}
  \begin{itemize}
  \item
  \end{itemize}
\end{frame}

\subsection{Moving uninflected lemma parts}
\begin{frame}
  \frametitle{Moving uninflected lemma parts}

\begin{figure}[h]
  \begin{center}
    \includegraphics[scale=0.7]{../bi.eps}
    \caption{Input bilingual FST (letter transitions compressed to single arcs)}
    \label{fig:bi-prefixed}
  \end{center}
\end{figure}

\begin{figure}[h]
  \begin{center}
    \includegraphics[scale=0.7]{../bi-prefixed.eps}
    \caption{Fully preprocessed bilingual FST; analyses \ana{take<vblex>\# out} and even \ana{take<vblex>+it<prn>\# out}    would be included after trimming on this}
    \label{fig:bi-prefixed}
  \end{center}
\end{figure}

\end{frame}

\subsection{Intersection}
\begin{frame}
  \frametitle{Intersection}

  \begin{figure}[h]
    \begin{center}
      \includegraphics[scale=0.7]{../mono.eps}
      \caption{Input monolingual FST}
      \label{fig:mono}
    \end{center}
  \end{figure}

  \begin{figure}[h]
    \begin{center}
      \includegraphics[scale=0.7]{../trimmed.eps}
      \caption{Trimmed monolingual FST}
      \label{fig:trimmed}
    \end{center}
  \end{figure}

\end{frame}

\subsection{\tool{lt-trim} in use}
\begin{frame}
  \frametitle{\tool{lt-trim} in use}
  \begin{itemize}
  \item
  \end{itemize}
\end{frame}

\section{Ending Dictionary Redundancy}
\begin{frame}
  \frametitle{Ending Dictionary Redundancy}
  \begin{itemize}
  \item
  \end{itemize}
\end{frame}

\section{Conclusion}
\begin{frame}
  \frametitle{Conclusion}
  \begin{itemize}
  \item
  \end{itemize}
\end{frame}

\section*{Acknowledgements}
\begin{frame}
  \frametitle{Acknowledgements}
  \begin{itemize}
  \item
  \end{itemize}
\end{frame}



\end{document}
