\documentclass[11pt]{article}
\usepackage[utf8x]{inputenc}
\usepackage{times}
\usepackage{natbib}
\usepackage{url}
\usepackage[small,bf]{caption}
\usepackage{latexsym}

\usepackage{hyperref}
\author{NN\\  X \\ X \\  X \\  {\tt \small   X@X} \And  NN \\  X \\  X \\  X \\    {\tt \small  X@X}}

\title{Trimming Language-Pair Independent FST's in Apertium}
% Avoiding Redundancy in Apertium FST's?

\begin{document}

\maketitle

\begin{abstract}
  A Finite State Transducers (FSTs) used as an analyser, whose output
  is input to another FST, may have entries that don't pass through
  the second FST. We describe the development of a tool to \emph{trim}
  such entries.
\end{abstract}

\section{Introduction}
Apertium is a rule-based machine translation platform which uses
Finite State Transducers for morphological analysis, bilingual
dictionary lookup and generation of surface forms.



\section{Discussion and outlook}
\section*{Acknowledgements}
Development was funded as part of the Google
Code-In\footnote{\href{http://code.google.com/soc/}{http://code.google.com/soc/}
} programme.


\bibliographystyle{apalike}
\bibliography{apertium}

\end{document}
