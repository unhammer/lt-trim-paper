\documentclass[11pt]{article}
\usepackage[utf8x]{inputenc}
\usepackage{times}
\usepackage{natbib}
\usepackage{url}
\usepackage[small,bf]{caption}
\usepackage{latexsym}

\usepackage{hyperref}
\author{NN\\  X \\ X \\  X \\  {\tt \small   X@X}}

\title{Trimming Language-Pair Independent FST's in Apertium}
% Avoiding Redundancy in Apertium FST's?

\begin{document}

\maketitle

\begin{abstract}
  A Finite State Transducers (FSTs) used as an analyser, whose output
  is input to another FST, may have entries that don't pass through
  the second FST. We describe the development of a tool to \emph{trim}
  such entries.
\end{abstract}

\section{Introduction}
Apertium is a rule-based machine translation platform which uses
Finite State Transducers for morphological analysis, bilingual
dictionary lookup and generation of surface forms.

Most translators created with Apertium use the \texttt{lttoolbox}
library for compiling XML dictionaries into binary FST's and for
processing text with such FST's. The bilingual dictionary transfers
not only lemmas, but tags, for example translating the tags noun
feminine into noun masculine.

If a word is in the analyser, but not in the bilingual translation
dictionary, the transfer module may apply the wrong tags . So until
very recently, each language pair / translator created with Apertium
would have its own set of monolingual data resources. This lead to a
lot of reduncancy, but was necessary in order to avoid
non-translatable strings getting half-way translated and possibly
messing up the context.


\section{Discussion and outlook}
\section*{Acknowledgements}
Development was funded as part of the Google
Code-In\footnote{\href{http://code.google.com/soc/}{http://code.google.com/soc/}
} programme.


\bibliographystyle{apalike}
\bibliography{apertium}

\end{document}
